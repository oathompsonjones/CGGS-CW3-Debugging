% Define the type of document.
\documentclass[12pt, a4paper]{article}

% Set a more sensible document size.
\usepackage[margin=0.75in]{geometry}

% Set the language.
\usepackage[british]{babel}
    \usepackage{csquotes}

% Prevent words from being split accross lines.
\tolerance=1\emergencystretch=\maxdimen\hyphenpenalty=10000\hbadness=10000

% Allows images to be included.
\usepackage{graphicx}
    \graphicspath{ {./images/} }

% Allow disabling of floating objects.
\usepackage{float}

% Allow the use of listings.
\usepackage[T1]{fontenc}
\usepackage[dvipsnames]{xcolor}
\usepackage{listings}
    \lstset{
        language=C++,
        basicstyle=\ttfamily,
        keywordstyle=[1]\color{Dandelion},
        keywordstyle=[2]\color{Cyan},
        keywordstyle=[3]\color{TealBlue},
        commentstyle=\color{Gray},
        numbers=left,
        numberstyle=\ttfamily,
        stepnumber=1,
        numbersep=5pt,
        tabsize=4,
        breaklines=true,
        captionpos=b,
        frame=single,
        escapechar=`,
        morekeywords=[1]{Eigen,JacobiSVD,MatrixXd,Matrix3d,Vector3d,Array3d,RowVector3d},
        morekeywords=[2]{transpose,row,singularValues,matrixU,matrixV,determinant,Identity,minCoeff,array},
        morekeywords=[3]{ComputeFullU,ComputeFullV,ComputeThinU,ComputeThinV},
    }

% Create some shortcut commands:
\newcommand{\removed}[1]{\colorbox{pink}{\vphantom{A}#1}}
\newcommand{\added}[1]{\colorbox{lime}{\vphantom{A}#1}}
\newcommand{\swapped}[2]{\removed{#1}\added{#2}}
\newcommand{\inline}[1]{\fbox{\texttt{#1}}}
\newcommand{\tab}[0]{\space\space\space\space}

% Sets the title.
\title{Computer Graphics: Geometry and Simulation\\Coursework 3: Debugging Geometry and Simulation}
\author{Oliver Jones (s2153980)}
\date{}

% Begin the document.
\begin{document}
\maketitle

\section*{Introduction}
    Throughout this report are a series of code snippets, highlighting the changes
        made to the original code to fix the bugs.
    The \removed{pink} sections have been removed from the original code, while the
        \added{green} sections have been added.

\section{As-Rigid-as-Possible Deformation}
    \subsection{}
        \begin{lstlisting}[caption={The \inline{0} and \inline{1} have been swapped.}, label={lst:a1}]
g.row(j) = (origV.row(E(j, `\swapped{0}{1}`)) - origV.row(E(j, `\swapped{1}{0}`))) * REdge;
                \end{lstlisting}

    \subsection{}
        \begin{lstlisting}[caption={The \inline{* W } was missing.}, label={lst:b1}]
MatrixXd rhs = d0I.transpose() `\added{* W }` * (g - d0B * constPositions);
          \end{lstlisting}

    \subsection{}
        \begin{lstlisting}[caption={Lines 3 and 5 have been removed. The \inline{for} loop is now closed 
            on line 6 instead of 22. \inline{P} and \inline{Q} on line 7 have been wapped. Lines 7 to 21
            are no longer indented.}, label={lst:c1}]
for (int k = 0; k < oneRings[j].size(); k++) {
    P.row(k) = origV.row(oneRings[j][k]) - origV.row(j);
`\removed{\tab{}P.row(k) = origV.row(oneRings[j][k]) - origV.row(j);}`
    Q.row(k) = currV.row(oneRings[j][k]) - currV.row(j);
`\removed{\tab{}Q.row(k) = currV.row(oneRings[j][k]) - currV.row(j);}`
`\added{\}}`
`\removed{\tab}`S = `\swapped{Q}{P}`.transpose() * `\swapped{P}{Q}`;
`\removed{\tab}`Eigen::JacobiSVD<Eigen::Matrix3d> svd(S, Eigen::ComputeFullU | Eigen::ComputeFullV);
`\removed{\tab}`Eigen::Matrix3d U = svd.matrixU();
`\removed{\tab}`Eigen::Vector3d Sigma = svd.singularValues();
`\removed{\tab}`Eigen::Matrix3d Vt = svd.matrixV().transpose();
`\removed{\tab}`Matrix3d currR = U * Vt;
`\removed{\tab}`if (currR.determinant() < 0.0) {
`\removed{\tab}`    // check where the smallest singular values falls
`\removed{\tab}`    int minValue, minIndex;
`\removed{\tab}`    minValue = Sigma.minCoeff(&minIndex);
`\removed{\tab}`    Matrix3d newSigma = Matrix3d::Identity();
`\removed{\tab}`    newSigma(minIndex, minIndex) = -1;
`\removed{\tab}`    currR = U * newSigma * Vt;
`\removed{\tab}`}
`\removed{\tab}`R[j] = currR;
`\removed{\}}`
            \end{lstlisting}

\section{Multi-body Rigid Simulation}
    \subsection{}
        Pairwise gravity was reversed.
        Two possible fixes:
        \begin{lstlisting}[caption={The \inline{+} and \inline{-} have been swapped.}, label={lst:d1}]
forces.row(i).array() = forces.row(i) `\swapped{-}{+}` forceDirection * gravityConstant / sqrDistance;
forces.row(j).array() = forces.row(j) `\swapped{+}{-}` forceDirection * gravityConstant / sqrDistance;
\end{lstlisting}
        \begin{lstlisting}[caption={The \inline{i} and \inline{j} have been swapped.}, label={lst:e1}]
RowVector3d forceDirection = spherePoses.row(`\swapped{j}{i}`) - spherePoses.row(`\swapped{i}{j}`);
\end{lstlisting}

    \subsection{}
        \begin{lstlisting}[caption={The \inline{2} has been replaced with \inline{4}.}, label={lst:f1}]
if (sqrDistance < `\swapped{2}{4}` * sqrRadius)
\end{lstlisting}

    \subsection{}
        \begin{lstlisting}[caption={The \inline{>} has been swapped with \inline{<}.}, label={lst:g1}]
if (sphereVelocities(i, 1) `\swapped{>}{<}` 0.0)
\end{lstlisting}

\end{document}
